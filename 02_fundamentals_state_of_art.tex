\chapter{Fundamentals and State of the Art}
En este capitulo se presentan los fundamentos necesarios para entender el trabajo en su totalidad y cual es el state of the art al momento del desarrollo de este trabajo.

\section{Task Management}

\section{State of the art}
\label{sec:stateOfTheArt}
Antes de desarrollar una aplicacion y definir sus funcionalidades, es vital analizar y estudiar el area al que se desea entrar. En este caso el area en cuestion es un sub-grupo del area de software de productividad. Segun Statista se espera que el area del software de productividad alcance en el 2022 ingresos de aproximadamente 73 billones de US dolares a nivel mundial y con proyecciones de un rapido y continuaodo crecimiento, alcanzando en el 2027 ingresos de aprox. 89 billones de US dolares, lo que significa un crecimiento de aprox. 25\% en los proximos 5 años. Esto siginifica que es un mercado muy grande y variado donde se ofrecen un sinfin de herramientas con diferentes enfoques.\\
Para hacerse una idea de que herramientas y productos para la organizacion personal de tareas hay actualmente en el mercado, se hizo un pequeño estudio informal. Analizando diferentes foros, como p.ej. Reddit, donde los usuarios buscaban aplicaciones y maneras para organizar sus actividades y tareas, se recopilo una pequeña lista con 45 diferentes herramientas. Basado en los features presentados por cada aplicacion o mecanismo se propone la siguiente categorizacion para organizar las diferentes herramientas basado en sus features:

\begin{description}
    \item [1. Digital Calendar] Ofrece unicamente una representacion digital de un calendario sin ningun otro tipo de visualizacion. Normalmente ofrece features como agregar tareas con fecha y hora, compartir dichas tareas, agregar recordatorios y notificaciones, etc. Algunos ejemplos de este tipo de aplicacion son Google Calendar, Apple Calendar y Tweek.
    \item [2. Simple text files] Una manera rapida y simple de mantener controlar y organizar tareas y actividades es utilizar un archivo de texto simple sin ningun tipo de funcionalidad avanzada o a veces sin siquiera la opcion de formatear el texto. Para esto se puede utilizar un sin fin de herramientas o aplicaciones, como por ejemplo Word, Notepad, Notepad++, etc. 
    \item [3. Native note-taking software] Hoy en dia la mayoria de los telefonos inteligentes y computadores tienen una aplicacion de notas nativa que viene pre-instalada. Estas aplicaciones tienen un enfoque mas general y permiten al usuario tomar notas de cualqier tipo de manera rapida y simple. Normalmente estas herramientas son muy simples y ofrecen funcionalidades basicas como la creacion y categorizacion de notas de texto simples. Algunas tienen funcionalidades un poco mas avanzadas como la creacion de check lists, la inclusion de imagenes y rich text editors pero no estan optimizadas para un use case especifico, por lo que se consideran su propia categoria.
    \item [4. Simple task manager] Estas aplicaciones estan tailor made para la creacion de listas y ofrecen unicamente esta funcionalidad. Unicamente incluyen features que enriquecen el proceso de creacion y manutencion de listas, como por ejemplo la categorizacion de entradas, notificaciones y recordatorios, due dates and times, agregar subtasks, generar tareas recurrentes, etc. Algunos ejemplos para este tipo de aplicacion son Google Keep, AnyList, Checkvist, etc.
    \item [5. Enhanced task manager] Los enhanced task manager son una expansion de los simple task managers, debido a que no solo incorporan las mismas funcionalidades basicas en cuanto a listas se refiere, sino brindan ademas funcionalidades mas avanzadas. Estas pueden ser, por ejemplo, la posibilidad de trabajar de manera colaborativa con otras personas u otros aspectos sociales, calendarios digitales ademas de las listas, Kanban boards, etc. Algunos ejemplos son Todoist, Google Tasks y Amie.
    \item [6. Project management software] Bajo esta categoria cae todo tipo de software originalmente disenado para el management de proyectos, pero que con el tiempo han sido adoptados para uso personal. Ejemplos de esto pueden ser Trello, Jira y Asana.
    \item [7. Highly customizable general purpose applications (HCGPA)] Esta ultima categoria representa software que no tiene ningun tipo de estructura fija y no prescribe al usuario cómo utilizarlo, sino que brinda a sus usuarios un lienzo en blanco y las herramientas para que puedan libremente construir las estructuras que necesiten. OneNote y Notion son dos herramientas que caen bajo esta categoria. 
\end{description}

\section{User Centered Design}
En este proyecto se siguio el enfoque del user centered design (UCD) para definir el producto final. Este proceso es iterativo y busca incluir desde el inicio y de manera constante a posibles futuros usuarios en el proceso de diseno y desarrollo de la aplicacion. De este modo, el desarrollo no se produce en una burbuja, aislado de feedback y opiniones externas y se evita la gran inversion de tiempo y recursos en un producto que posiblemente nadie quiera o sea recibido de manera diferente a la esperada. Abrir el proyecto desde el inicio a otros ojos y diferentes perspectivas hace que sea obligatorio generar diferentes iteraciones basado en el feedback recbido, antes de comprometerse con el producto final. Esto reduce la cantidad de asunciones que debe hacerse durante el desarrollo y basa las decisiones en las necesidades y deseos reales expresados por los participantes. Idealmente, esto permite la creacion de una herramienta mas madura y cercana a un producto final. 
Por el otro lado, este proceso al ser iterativo y poderse aplicar en cualquier momento del ciclo de vida del producto, tambien significa que realmente no tiene un final claro. Dadas las limitaciones de tiempo de este proyecto, se eligieron diferentes herramientas del proceso de UCD que permitieran evaluar y recopilar datos y opiniones sobre los posibles futuros usuarios, en un plazo de tiempo manejable que también concediera tiempo suficiente para el desarrollo real del software.
\\
Se utilizaron diferentes herramientas para recopilar, validar y aplicar las opiniones, deseos y costumbres de los usuarios. Primero se realizo un focus group con 6 a 9 participantes, como es usualmente recomendado \cite{}. Con este focus group se buscaba obtener una idea de como los participantes organizaban sus tareas, que herramientas usaban, con que frecuencia lo hacian, etc. Esta informacion permitiria definir mejor hacia que direccion se debia mover la aplicacion y donde podia haber un posible espacio para nuevas funcionalidades. 
Luego se realizo un survey donde se presentaro los datos recopilados en el focus group a un mayor publico para validar y verificar si los patrones de conducta presentados en el focus group tambien se veian reflejados cuando se aumentaba el tamano del grupo consultado. Con esto se buscaba confirmar que el feedback y las ideas recopiladas con la ayuda del focus group no fueran solo validos para ese pequeno grupo de gente sino tambien para el publico en general. Ademas se de validar la informacion, se buscaba recopilar un poco de informacion sobre los habitos de los participantes a la hora de organizar sus tareas.
Una vez que se recopilo y valido la informacion sobre los comportamientos y costumbres de los usuarios se prosiguio a aplicar los conocimientos recopiladoy y desarrollar un primer prototipo rapido de papel. Con este se podia entonces realizar una evaluacion heuristica, donde se buscaba identificar posibles problemas de usabilidad en la solucion propuesta, eliminando asi posibles problemas aun antes de haber desarrollado la aplicacion, haciendo el proceso mas eficaz y economico.
Luego se genero un simple mockup digital donde no solo se querian mejorar los puntos planteados en la evaluación anterior sino tambien integrar los temas de usability y ux. Estso no fueron el objetivo principal del prototipo de papel y se dejaron un poco de lado, debido a que se trataba de una propuesta rápida y aproximada. Este prototipo digital fue el que se utilizo como plantilla a la hora de desarrollar la aplicacion. 
Por ultimo, una vez que se hubo desarrollado la aplicacion se llevo a cabo una evaluacion de usabilidad, para conseguir posibles problemas o puntos criticos antes del lanzamiento de la aplicacion. Con este paso se cierra no solo el proceso de ucd durante el desarrollo sino tambien este proyecto. 
